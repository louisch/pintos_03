\documentclass[11pt]{article}

\usepackage{fullpage}

\begin{document}

\title{Task 0 Answers and Design Document}
\author{Anton Alenov \and Adam Gestwa \and Alan Han \and Louis Chan}

\maketitle

\section{Answers to Questions}

\begin{enumerate}

\item Which Git command should you run to retrieve a copy of your group's shared
Pintos repository in your local directory?

\emph{git clone git@gitlab.doc.ic.ac.uk:lab1415\_spring/pintos\_03.git}

or

\emph{git clone https://gitlab.doc.ic.ac.uk/lab1415\_spring/pintos\_03.git}

\item Why is using the \emph{strcpy()} function to copy strings usually a bad
idea?
(\emph{Hint: identify the problem, give details and discuss possible
solutions.})

\emph{strcpy(dst, src)} can cause a buffer overflow. This happens if dst has
less space allocated to it than the size of \emph{src}.

Furthermore, while \emph{strcpy()} expects \emph{src} to be NUL-terminated, it
does not actually check that it is and continues to copy bytes until it either
finds a NUL-terminator (resulting in an out-of-bounds read and potentially
leading to a buffer overflow) or runs out of memory, leading to segmentation
faults.

This can be solved by checking the size of \emph{dst} before using \emph{strcpy}
every time and systematically making sure that \emph{src} is NUL-terminated by
inserting a NUL-terminator into the string.

Alternatively, \emph{strlcpy} can also be used which requires the programmer to
pass in the size of dst, and performs this check internally.

\item Explain how thread scheduling in Pintos currently works in less than 300
words. Include the chain of execution of function calls.
(\emph{Hint: we expect you to at least mention which functions participate in a
context switch, how they interact, how and when the thread state is modified and
the role of interrupts.})

\emph{thread\_block()} sets the current thread to THREAD\_BLOCKED and then calls
\emph{schedule()} to switch threads.

\emph{thread\_yield()} sets the current thread to THREAD\_READY adding it to the
end of the queue (\emph{ready\_queue}) and then calls \emph{schedule()} to switch
threads.

\emph{thread\_exit()} sets the current thread to THREAD\_DYING and then calls
\emph{schedule()} which then kills the current thread.

When \emph{schedule()} is called it gets the currently running thread using
\emph{running\_thread()} and the next thread to run using
\emph{next\_thread\_to\_run()}.

Interrupts must be turned off when switching threads. The function asserts this
via the function \emph{intr\_get\_level()}.

It checks that the current thread has been set to a status which is not
\emph{THREAD\_RUNNING}, before then asserting that the next thread to run is
valid using \emph{is\_thread()}.

If the next thread is not equal to the current thread. It switches to it using
\emph{switch\_threads()} and sets the variable previous to the current
thread. Finally, \emph{thread\_schedule\_tail()} is called on previous.

\emph{thread\_schedule\_tail()} finishes the switch, performing several tasks to
complete it. If a switch happened, and the previously current thread has status
\emph{THREAD\_DYING}, then it is here that it gets killed properly.

\item Explain the property of reproducibility and how the lack of
reproducibility will affect debugging.

In debugging, the reproducibility of an error refers to the ability to
successfully reproduce the given error given a certain set of input.

Naturally, in the ideal case, one would be able to reproduce an error every time
by following specific steps. However, in certain cases, notably in
multi-threading, errors arise from events that are not fully dependant on user
input.

In theory, it is impossible to guarantee with tests alone that such bugs do not
exist or have been corrected accordingly in a program, as they will not
necessarily surface during the tests. Overall, lack of reproducibility leads to
longer debugging times as just running the test suite once is no longer enough
to ensure that the code is bug free and other measures, such as stress testing
or formal models need to be used.

\item How would you print an unsigned 64 bit \emph{int}?
(Consider that you are working with C99). Don't forget to state any inclusions
needed by your code.

The code would look something like this:

\begin{verbatim}
/* necessary inclusions */
#include <stdint.h>;
#include <inttypes.h>;

/* variable declaration */
uint64_t var = xxx;

/*
 * print statement
 * several formats are available, for instance:
 * decimal (PRId64), octal (PRIo64), hexadecimal (PRIx64).
 */
printf("%"PRId64, var);
\end{verbatim}

\item Describe the data structures and functions that locks and semaphores in
Pintos have in common. What extra property do locks have that semaphores do not?

Locks and semaphores present a way to control access to a resource. Both allow
threads to acquire/release the resource (through
\emph{lock\_acquire()}/\emph{lock\_release()} or
\emph{sema\_down()}/\emph{sema\_up()} respectively), and for make threads wait
on a resource if it is not immediately available (through \emph{sema\_down()} or
\emph{lock\_acquire()}). In addition, \emph{sema\_try\_down()} and
\emph{lock\_try\_acquire()} are methods that try to acquire a resource,
returning whether they succeeded or not.

The primary data structure which allows this functionality is the semaphore
struct. It contains an unsigned value field which indicates whether a resource
is available, and a list of threads which are blocked and waiting on the
resource. Presumably to avoid duplication, the lock struct also contains a
semaphore, thus inheriting those two fields.

However, locks can only be released by the thread they were acquired by, in
contrast to semaphores, which can be upped by any thread. The method
\emph{lock\_held\_by\_current\_thread()} can be used to check whether a
particular thread is currently holding the lock.

\item In Pintos, a thread is characterized by a struct and an execution stack.
What are the limitations on the size of these data structures? Explain how this
relates to stack overflow and how Pintos identifies it.

The struct and execution stack are stored in their own 4 kB page with the
stack located above the struct. The size of the execution stack is restricted by
the struct (size of execution stack = 4kB - execution stack).

The size of the struct should remain at a minimum to allow the stack as
much space as possible to grow. The smaller the stack the greater probability of
a stack overflow.

Pintos detects stack overflows by checking whether the \emph{magic} member of
the running thread's \emph{struct thread} has changed from
\emph{THREAD\_MAGIC}. The \emph{magic} member is located at the bottom of the
struct and therefore will be the first data member to be overwritten by the
stack overflow.

\item If test \emph{src/tests/devices/alarm-multiple} fails, where would you
find its output and result logs? Provide both paths and file names.
(\emph{Hint : you might want to run this test to find out.})

\emph{pintos\_03/src/devices/build/tests/devices/alarm-multiple.output}

\emph{pintos\_03/src/devices/build/tests/devices/alarm-multiple.result}

\end{enumerate}

\section{Design Document}

\subsection{Data Structures}
\paragraph{A1:}
Copy here the declaration of each new or changed ‘struct’ or ‘struct’ member,
global or static variable, ‘typedef’, or enumeration. Identify the purpose of
each in 25 words or less.
\subsubsection{Code}

\paragraph{(src/devices/timer.c):}

\begin{verbatim}
+static struct list sleepy_threads;
\end{verbatim}

\paragraph{(src/threads/threads.h):}

\begin{verbatim}
struct thread
  {
     ...
+    /* Owned by devices/timer.c */
+    int sleep_time;                     /* Time until wake-up */
+    struct list_elem sleepy_elem;       /* Sleep list element */
     ...
  };

\end{verbatim}

\subsubsection{Purpose}
\paragraph{static struct list sleepy\_threads:}
A list of all currently sleeping threads.

\paragraph{struct thread}
Added 2 fields to regulate sleeping functionality.

\subparagraph{int sleep\_time:}
Used for keeping track of the number of ticks remaining to sleep for.
\subparagraph{list\_elem sleepy\_elem:}
Used as a pointer into the list of sleeping threads. When iterating over
sleepy\_threads this is used to get the thread that contains each list\_elem.

\subsection{Algorithms}
\paragraph{A2:}
Briefly describe what happens in a call to timer\_sleep(), including the actions
performed by the timer interrupt handler on each timer tick.

The timer\_sleep() method sets the current thread's sleep\_time field to ticks
to sleep for \"ticks\" and pushes it to the back of the sleepy\_threads
list. List access is synchronised by disabling interrupts before writing the
list and re-enabling them after blocking the thread.

The method timer\_interrupt() is called on every hardware tick. If
sleepy\_threads is not empty, it calls timer\_wake\_threads() to iterate through
every entry on the list, decrementing their remaining sleep time and unblocking
threads whose time is less than or equal to 0.

\paragraph{A3:}
What steps are taken to minimize the amount of time spent in the timer interrupt
handler?

Timer sleeps avoids blocking threads for negative sleep times, essentially
preventing unnecessary thread blocking. Furthermore, within the interrupt
handler, we check if sleepy\_list is empty before attempting to iterate over the
list and wake threads.

\subsection{Synchronisation}
\paragraph{A4:}
How are race conditions avoided when multiple threads call timer\_sleep()
simultaneously?

We disable interrupts while adding to sleepy\_threads and blocking the
thread. It is henceforth impossible for two timer\_sleep() operations to access
the list or block their threads asynchronously, as disabling interrupts
essentially combines these two operations into a single atomic operation.

\paragraph{A5:}
How are race conditions avoided when a timer interrupt occurs during a call to
timer\_sleep()?

There are two parts of timer\_sleep() which require synchronisation: adding to
sleepy\_threads and calling thread\_block(). Moreover, an interrupt between
these operations could lead to attempting to unblock threads that have been
added to sleepy\_threads, but not blocked yet. Finally, thread\_block() requires
interrupts to be disabled before it is called.

We avoid both of the race conditions and fulfil the thread\_block() requirement
by disabling interrupts for both operations. Note that it is not possible to
block the thread before adding it to the list, as then the list add operation
will never run.

\subsection{Rationale}
\paragraph{A6:}
Why did you choose this design? In what ways is it superior to another design
you considered?

In our re-implementation of timer\_sleep() we opted for using a list shared
between the thread calling timer\_sleep() and the interrupt handler, which we
use to wake up the threads. We disable interrupts in order to synchronise access
to this list.

During the development of this solution, we have considered using pre-existing
synchronisation primitives (locks and semaphores), however we eventually came to
the conclusion that neither of these were fully appropriate.

Locks were suggested for controlling access to the sleeping thread list,
however, as it is impossible to acquire locks inside the interrupt handler, they
could not prevent the handler from concurrently reading or modifying the list.

Semaphores are functionally similar to our current implementation (both our code
and sema\_down() disable interrupts to add the current thread to a list before
blocking it), however they do not present a convenient way to keep track of
multiple sleeping threads at a time with a single semaphore.

Creating a semaphore per sleeping thread is a possible alternative, however, in
order to keep track of these semaphores, a list is required. Not only would this
bring up the same list synchronisation issues as our current implementation, but
we would need a separate struct to bind semaphores to list\_elems.

\end{document}
