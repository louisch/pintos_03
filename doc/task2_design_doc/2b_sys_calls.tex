\section{System Calls}
\subsection{Data Structures}
\paragraph{B1:} % (10 marks)
Copy here the declaration of each new or changed `struct' or `struct' member, global or static variable, `typedef', or enumeration.  Identify the purpose of each in 25 words or less.

\subparagraph{process.c global static variables}

\begin{verbatim}
/* The lock for the below hash table */
static struct lock process_info_lock;
/* Maps pids to process_infos. Also serves to keep track of all
   processes that exist. */
static struct hash process_info_table;
\end{verbatim}

The `process\_info\_table' keeps tracks of process information. Each process's
information is stored in a `process\_info' struct (see below). Hashes by pid.

\begin{verbatim}
/* Used for allocating pids. */
static struct lock next_pid_lock;
\end{verbatim}

This is only used for allocating pids. This is to lock a static variable that
is local to the function that allocates the pids.

\subparagraph{process\_info}

\begin{verbatim}
/* Data for a process used for syscalls. */
typedef struct process_info
{
  /* Process ID. */
  pid_t pid;
  /* ID of the thread owned by the process. */
  tid_t tid;

  /* Exit status of process. */
  int exit_status;
  /* Lock to synchronise access to children hashtable. */
  struct lock children_lock;
  /* Tracks the process's children, running or terminated,
     which have not been waited on. */
  struct hash children;
  /* Pointer to parent's child_info struct. */
  struct child_info *parent_child_info;

  /* For placing process_info in hash table mapping pids to process_info. */
  struct hash_elem process_elem;

  unsigned fd_counter;
  struct hash open_files;

} process_info;
\end{verbatim}

`process\_info' represents various pieces of information attached to a process
that needs to be saved.

\begin{verbatim}
  /* Process ID. */
  pid_t pid;
  /* ID of the thread owned by the process. */
  tid_t tid;
\end{verbatim}

The pid of a process unique identifies the process, and the tid identifies the
thread owned by the process.

\begin{verbatim}
  /* Exit status of process. */
  int exit_status;
\end{verbatim}

Used by `syscall\_exit', this is the exit status of the process, set by the exit
system call. It is initialized to -1.

\begin{verbatim}
  /* Lock to synchronise access to children hashtable. */
  struct lock children_lock;
  /* Tracks the process's children, running or terminated,
     which have not been waited on. */
  struct hash children;
\end{verbatim}

The `children' hash tracks the children of a process, using the pid to hash
them.

\begin{verbatim}
  /* Pointer to parent's child_info struct. */
  struct child_info *parent_child_info;
\end{verbatim}

This is a pointer to this thread's parent's `child\_info' struct (see below).

\begin{verbatim}
  /* For placing process_info in hash table mapping pids to process_info. */
  struct hash_elem process_elem;
\end{verbatim}

This is used to place the process\_info into the static hash table
`process\_info\_table'.

\subparagraph{child\_info}

\begin{verbatim}
typedef struct child_info
{
  /* Child thread's ID, used when checking if a child belongs to a parent. */
  tid_t tid;
  /* Exit status of thread. */
  int exit_status;
  /* Indicates whether the child process is still running. */
  bool running;
  /* Pointer to parent wait semaphore. Is NULL if the parent is not waiting. */
  struct semaphore *parent_wait_sema;
  /* Pointer to child process_info. */
  process_info *child_process_info;
  /* Hash elem to be placed into process_info's children hash. */
  struct hash_elem child_elem;

} child_info;
\end{verbatim}

\paragraph{B2:} % (5 marks)
Describe how file descriptors are associated with open files. Are file descriptors unique within the entire OS or just within a single process?

\subsection{Algorithms}
\paragraph{B3:} % (5 marks)
Describe your code for reading and writing user data from the kernel.

\paragraph{B4:} % (5 marks)
Suppose a system call causes a full page (4,096 bytes) of data to be copied from user space into the kernel.  What is the least and the greatest possible number of inspections of the page table (e.g. calls to pagedir\_get\_page()) that might result?  What about for a system call that only copies 2 bytes of data?  Is there room for improvement in these numbers, and how much?

\paragraph{B5:} % (5 marks)
Briefly describe your implementation of the ``wait'' system call and how it interacts with process termination.

\paragraph{B6:} % (5 marks)
Any access to user program memory at a user-specified address can fail due to a bad pointer value.  Such accesses must cause the process to be terminated.  System calls are fraught with such accesses, e.g. a ``write'' system call requires reading the system call number from the user stack, then each of the call's three arguments, then an arbitrary amount of user memory, and any of these can fail at any point.  This poses a design and error-handling problem: how do you best avoid obscuring the primary function of code in a morass of error-handling?  Furthermore, when an error is detected, how do you ensure that all temporarily allocated resources (locks, buffers, etc.) are freed?  In a few paragraphs, describe the strategy or strategies you adopted for managing these issues.  Give an example.

\subsection{Synchronization}

\paragraph{B7:} % (5 marks)
The ``exec'' system call returns -1 if loading the new executable fails, so it cannot return before the new executable has completed loading.  How does your code ensure this?  How is the load success/failure status passed back to the thread that calls ``exec''?

\paragraph{B8:} % (5 marks)
Consider parent process P with child process C.  How do you ensure proper synchronization and avoid race conditions when P calls wait(C) before C exits?  After C exits?  How do you ensure that all resources are freed in each case?  How about when P terminates without waiting, before C exits?  After C exits?  Are there any special cases?

\subsection{Rationale}

\paragraph{B9:} % (5 marks)
Why did you choose to implement access to user memory from the kernel in the way that you did?

\paragraph{B10:} % (5 marks)
What advantages or disadvantages can you see to your design for file descriptors?

\paragraph{B11:} % (5 marks)
The default tid\_t to pid\_t mapping is the identity mapping. If you changed it, what advantages are there to your approach?
