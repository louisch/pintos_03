\section{Page Table Management}

\subsection{Data Structures}

% Copy here the declaration of each new or changed `struct' or
% `struct' member, global or static variable, `typedef', or
% enumeration.  Identify the purpose of each in 25 words or less.

\subsubsection*{threads/thread.h}

We added two fields into struct thread.

\begin{verbatim}
struct thread
  {
...
    struct supp_page_table supp_page_table; /* Supplementary Page Table. */
    void *stack_bottom; /* Address of page at bottom of allocated stack. */
...
  };
\end{verbatim}

supp\_page\_table is the supplementary page table for that thread (see below for
more details).

stack\_bottom keeps track of the address of the bottom of the stack we have
allocated so far. This is used for the stack growth algorithm.

\subsubsection*{vm/supp\_page.h}

\begin{verbatim}
/* The supplementary page table keeps track of additional information on each
   page that the page table (in pagedir.h) cannot.

   This is used during page faults, to find out where the data for a virtual
   page is, so that the data can be read into memory, and then mapped in the
   page table. Pages may have been swapped out into swap space, or they may
   simply be in a file, or in the case of a stack growth, the page may not exist
   anywhere yet, and it simply needs to be created.

   It is organised in terms of segments. For the supplementary page table,
   segments are viewed as consecutive sequences of virtual pages. For example,
   the code segment of an executable is mapped as a consecutive sequence of
   virtual pages in virtual memory. */
struct supp_page_table
  {
    struct list segments; /* Entries in this table. */
  };
\end{verbatim}

supp\_page\_table stores the segments of the supplementary page table, and
represents a thread's supplementary page table as a whole.

\begin{verbatim}
/* This is where the data for each segment that the user wants to load into
   memory is stored. Information for reading pages from this segment into
   memory is kept.

   Segments may have originated from a file (they are read from a file into
   memory). If they did not, then the file_data field will be set to NULL.

   Pages that lie in file segments are read from the file itself, using the
   file_data, while pages from non-file segments are simply zeroed out on
   creation.

   Pages which have been mapped already keep track of whether they are in swap
   or not. */
struct supp_page_segment
  {
    struct list_elem supp_elem; /* For placing this into a supp_page_table. */
    void *addr; /* The virtual user address this segment begins at. */

    struct supp_page_file_data *file_data; /* File data */
    struct hash mapped_pages; /* Previously mapped pages. */

    /* Other properties */
    bool writable; /* Whether this segment is writable or not. */
    uint32_t size; /* The size of the segment. */
  };
\end{verbatim}

supp\_page\_segment represents a contiguous sequence of pages in virtual memory.
It keeps track of data that is shared between all the pages of that segment.

\begin{verbatim}
/* Data pertaining to a segment that has data which exists in a file. */
struct supp_page_file_data
  {
    struct file *file; /* The file that this page is read from. NULL if not a file. */
    uint32_t offset; /* Offset from which segment starts from, in file. */
    /* This is the number of bytes to read from the file, starting from the
       offset. If the size of the segment is larger than read_bytes, then the
       remaining bytes are zeroed out. (Of course, this is all read lazily, page
       by page, in whatever order the user decides to access the data). */
    uint32_t read_bytes;
    bool is_mmapped;
  };
\end{verbatim}

supp\_page\_file\_data is additional data that needs to be stored for segments
that need to be read in from files.

\begin{verbatim}
/* Represents a page that will be installed in the pagedir.
   This is used to keep track of where to read a mapped page back in if it page
   faults. Its location may be in swap, or in a file, or simply nowhere (if a
   clean stack page pages-out, for instance, it is not read to swap, so a new
   stack page will be zeroed out and installed). */
struct supp_page_mapping
{
  struct hash_elem mapping_elem; /* For placing this in supp_page_segment. */
  struct supp_page_segment *segment; /* A pointer back to the segment that contains this. */
  void *uaddr; /* The virtual user address this page begins at. */
  slot_no swap_slot_no; /* Slot number of this page in swap, if it lies in swap. */
  struct lock eviction_lock;
};
\end{verbatim}

supp\_page\_mapping is a struct which is created each time a newly faulting page
is mapped. This is for keeping track of if the page gets swapped out during its
lifetime.

\subsection{Algorithms - Frame Locating}

% In a few paragraphs, describe your code for locating the frame,
% if any, that contains the data of a given page.

During a page fault, we find which segment a page is in (this is necessary
because the segment struct contains metadata which is required for this locating
algorithm), and then we check whether the page has previously been mapped. If it
has, then we will find a struct that tells us whether the page lies in swap or
not. It also tells us where in swap it lies if the page does currently reside in
swap.

If this struct cannot be found, then we can infer which page of the file the
faulting page resides in, from data stored in the segment (a struct file
pointer, the offset into the file that the segment begins at, etc.).

When requesting a new frame, we use the unchanged palloc\_get\_page function to
locate a free frame to use. If no frame is returned, then we evict some frame to
use (more details on this further down).

\subsection{Algorithms - Aliases of Accessed and Dirty bits}

% How does your code coordinate accessed and dirty bits between
% kernel and user virtual addresses that alias a single frame, or
% alternatively how do you avoid the issue?

We simply check the accessed and dirty bits of both, summing the number of bits
set for all aliases of a frame.

\subsection{Synchronization - Concurrent frame requests}

% When two user processes both need a new frame at the same time,
% how are races avoided?

The frame table has a lock, which is acquired inside the request\_frame
function when it does its work, and released before it returns. If two user
processes request a frame at the same time, then the second call blocks
until the first is finished.

\subsection{Rationale - Virtual-to-Physical Mappings}

% Why did you choose the data structure(s) that you did for
% representing virtual-to-physical mappings?

We chose to keep track of data about segments of user virtual pages,
corresponding to the code and data segments of the user executable, but also
allowing us to group the pages of every memory-mapped file, and the pages of the
stack together. This means that any data that applies to all pages within a
segment is shared among them. The code would perhaps be simpler if we had
organised purely by pages, and had copies of this data for every page in a
segment, but this would be hugely memory inefficient.

The only additional requirement is that segments must keep a hash table of the
pages that have been mapped previously to frames within it.

Pages are either on a file, in swap, in memory, or simply not mapped (as in the
case of paged-out un-dirty stack pages for instance) and during a page fault, we
only need to know whether it is in a file, in swap, or in no particular location
(since it clearly isn't in memory). Since only pages that have already been
mapped can be swapped out, it is sufficient to simply store some small amount of
data on mapped pages to indicate whether they are in swap, and if they are,
where they are in swap.

Using a list to keep track of segments means iterating over the list during each
page fault, which is somewhat slow. It works fast enough in practice however,
and we didn't want to use too much time implementing a faster data structure.
