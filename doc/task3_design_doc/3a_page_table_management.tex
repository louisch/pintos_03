\section{Page Table Management}

\subsection{Data Structures}

% Copy here the declaration of each new or changed `struct' or
% `struct' member, global or static variable, `typedef', or
% enumeration.  Identify the purpose of each in 25 words or less.

\subsubsection*{threads/thread.h}

We added two fields into struct thread.

\begin{verbatim}
struct thread
  {
...
    struct supp_page_table supp_page_table; /* Supplementary Page Table. */
    void *stack_bottom; /* Address of page at bottom of allocated stack. */
...
  };
\end{verbatim}

supp\_page\_table is the supplementary page table for that thread (see below for
more details).

stack\_bottom keeps track of the address of the bottom of the stack we have
allocated so far. This is used for the stack growth algorithm.

\subsubsection*{vm/supp\_page.h}

\begin{verbatim}
/* The supplementary page table keeps track of additional information on each
   page that the page table (in pagedir.h) cannot.

   This is used during page faults, to find out where the data for a virtual
   page is, so that the data can be read into memory, and then mapped in the
   page table. Pages may have been swapped out into swap space, or they may
   simply be in a file, or in the case of a stack growth, the page may not exist
   anywhere yet, and it simply needs to be created.

   It is organised in terms of segments. For the supplementary page table,
   segments are viewed as consecutive sequences of virtual pages. For example,
   the code segment of an executable is mapped as a consecutive sequence of
   virtual pages in virtual memory. */
struct supp_page_table
  {
    struct list segments; /* Entries in this table. */
  };
\end{verbatim}

supp\_page\_table stores the segments of the supplementary page table, and
represents a thread's supplementary page table as a whole.

\begin{verbatim}
/* This is where the data for each segment that the user wants to load into
   memory is stored. Information for reading pages from this segment into
   memory is kept.

   Segments may have originated from a file (they are read from a file into
   memory). If they did not, then the file_data field will be set to NULL.

   Pages that lie in file segments are read from the file itself, using the
   file_data, while pages from non-file segments are simply zeroed out on
   creation.

   Pages which have been mapped already keep track of whether they are in swap
   or not. */
struct supp_page_segment
  {
    struct list_elem supp_elem; /* For placing this into a supp_page_table. */
    void *addr; /* The virtual user address this segment begins at. */

    struct supp_page_file_data *file_data; /* File data */
    struct hash mapped_pages; /* Previously mapped pages. */

    /* Other properties */
    bool writable; /* Whether this segment is writable or not. */
    uint32_t size; /* The size of the segment. */
  };
\end{verbatim}

supp\_page\_segment represents a contiguous sequence of pages in virtual memory.
It keeps track of data that is shared between all the pages of that segment.

\begin{verbatim}
/* Data pertaining to a segment that has data which exists in a file. */
struct supp_page_file_data
  {
    struct file *file; /* The file that this page is read from. NULL if not a file. */
    uint32_t offset; /* Offset from which segment starts from, in file. */
    /* This is the number of bytes to read from the file, starting from the
       offset. If the size of the segment is larger than read_bytes, then the
       remaining bytes are zeroed out. (Of course, this is all read lazily, page
       by page, in whatever order the user decides to access the data). */
    uint32_t read_bytes;
    bool is_mmapped;
  };
\end{verbatim}

supp\_page\_file\_data is additional data that needs to be stored for segments
that need to be read in from files.

\begin{verbatim}
/* Represents a page that will be installed in the pagedir.
   This is used to keep track of where to read a mapped page back in if it page
   faults. Its location may be in swap, or in a file, or simply nowhere (if a
   clean stack page pages-out, for instance, it is not read to swap, so a new
   stack page will be zeroed out and installed). */
struct supp_page_mapping
{
  struct hash_elem mapping_elem; /* For placing this in supp_page_segment. */
  struct supp_page_segment *segment; /* A pointer back to the segment that contains this. */
  void *uaddr; /* The virtual user address this page begins at. */
  slot_no swap_slot_no; /* Slot number of this page in swap, if it lies in swap. */
  struct lock eviction_lock;
};
\end{verbatim}

supp\_page\_mapping is a struct which is created each time a newly faulting page
is mapped. This is for keeping track of if the page gets swapped out during its
lifetime.

\subsection{Algorithms - Frame Locating}

% In a few paragraphs, describe your code for locating the frame,
% if any, that contains the data of a given page.

During a page fault, we find which segment a page is in. We locate a
supp\_page\_segment for the faulting address, which will succeed as long as the
faulting address is valid.

We always start by requesting a frame from the frame table to use.  When
requesting a new frame, we use the unchanged palloc\_get\_page function to
locate a free frame to use. If no frame is returned, then we evict some frame to
use (more details on this further down). Here, there is the possibility for
checking whether the page should be all zeroes, and then fetching some special
shared read-only frame that was previously filled with zeroes if this is the
case.  However, we do not implement this.

We then we try to locate a supp\_page\_mapping for the page. This struct
contains data that tells us whether the page lies in swap or not. It also tells
us where in swap it lies if the page does currently reside in swap. If it does
lie in swap, then we read the page into the frame we requested.

If this struct cannot be found for the faulting address, then the page is
definitely not in swap. Either way, if the page does not lie in swap, then we
check whether the segment is a file segment. The supp\_page\_segment holds a
pointer to its supp\_page\_file\_data, which is NULL if it is not a file
segment. The file data struct keeps track of the offset into the file that the
segment starts at, the number of bytes that need to be read from the file, and
the size of the segment as a whole in bytes. After rounding the faulting address
down to a page boundary, we have all the information required to read a faulting
page of a file segment into the frame requested earlier, and to install it into
the pagedir.

Otherwise, the page is a new non-file segment, and is a page that we have not
previously mapped before. The stack is the only segment in this category.  In
this case, we just zero out the frame we requested earlier.

\subsection{Algorithms - Aliases of Accessed and Dirty bits}

% How does your code coordinate accessed and dirty bits between
% kernel and user virtual addresses that alias a single frame, or
% alternatively how do you avoid the issue?

We simply check the accessed and dirty bits of both, summing the number of bits
set for all aliases of a frame. The frame table keeps track of the
supp\_page\_mapping that has been mapped to each frame, which in turn contains
the virtual user address of the page. It also keeps track of the virtual kernel
address of the frame, so it can check the accessed and dirty bits of both
addresses.

\subsection{Synchronization - Concurrent frame requests}

% When two user processes both need a new frame at the same time,
% how are races avoided?

The frame table has a lock, which is acquired inside the request\_frame
function when it does its work, and released before it returns. If two user
processes request a frame at the same time, then the second call blocks
until the first is finished.

\subsection{Rationale - Virtual-to-Physical Mappings}

% Why did you choose the data structure(s) that you did for
% representing virtual-to-physical mappings?

We chose to organise by segments. The supp\_page\_segments represent the code
and data segments of the user executable quite nicely, but can also be used to
represent any contiguous sequence of pages, so we register the stack, as well as
each memory-mapped file, as a segment.

This means that any data that applies to all pages within a segment is shared
among them. The code would perhaps be simpler if we had organised purely by
pages, and had copies of this data for every page in a segment, but this would
be hugely memory inefficient. Data that needs to be shared, for example, is the
supp\_page\_file\_data that applies to all pages within a segment.

The only additional requirement is that segments must keep a hash table of the
pages that have been mapped previously to frames within it, so that if the pages
are swapped out this could be recorded. A hash table is a good choice here, as
it provides constant time lookup.

Using a list to keep track of segments means iterating over the list during each
page fault, which is somewhat slow. We cannot simply use the hash table
mentioned above directly, as the shared data within the segment is required. It
works fast enough in practice however, and we didn't want to use too much time
implementing a faster data structure.

Keeping track of some data for every single mapping that each process makes would
require a lot of memory if multiple processes uses a lot of memory.
% TODO: Would creating a hash table from uaddr to swap_no in swap table solve this problem?

We have a frame table data structure that performs the work of palloc'ing pages
and evicting pages if required. It has a hash table which provides quick lookup
of a frame via kernel virtual address, when we need to free a specific frame. It
also has a list of frames, so that we can treat the frames as a circular queue
for our eviction algorithm. This makes the code for creating and destroying frames
slightly more difficult to write, as two data structures need to be updated, but
this is a trade-off we were willing to make for the increased speed.

The swap table keeps track of free swap slots by storing ranges of free slots into a list, ordered by the lowest starting slot number. This has the advantage of being memory-efficient and quite scalable for larger swap disks. Because the ranges are stored in an ordered list, allocating a free swap slot is done in constant time by taking the first free slot of the first free range, and decrementing the range by one (and removing it if necessary).

Freeing swap slots is more tricky as we need to iterate through the the existing ranges to find the place of insertion, before either creating a new range, extending an existing range or merging two ranges into one. However, the number of ranges should on average be quite small in comparison

Swap table
	uses ranges of free slots as a memory-efficient way of marking free slots
	Because free swap slot ranges are stored in an ordered list (with the lowest start slot first), allocating a free swap slot is very fast - a simple matter of taking the first free slot of the first free range.
	Freeing a slot is more complex as you need to iterate through existing ranges, before either creating new range, extending an existing range or merging 2 existing ranges into one. However this is an acceptable (and necessary) trade-off for the amount of memory saved, and the .
	Bitmaps are an alternative solution. However, they make searching for the next free slot slower as you would have to iterate through the bitmap, and the number of bitmap cells to iterate through is likely higher than the number of ranges needed to iterate through for range insertion.
	It is theoretically possible for the ranges to become very fragmented. However, in practice this is unlikely because the next free slot is always the lowest available slot number, thus filling in the gaps. It is possible to implement a limit on the number of ranges and ``defragment'' the swap disk should this limit be exceeded. However, this is likely a bad idea in practice because disk access is many orders of magnitude slower than memory access, and it has not been implemented for this reason.

% TODO: Talk about data structures used in swap table
% TODO: Talk about data structures used in memory mapped files
