\section{Paging To and From Disk}

\subsection{Data Structures}

% Copy here the declaration of each new or changed `struct' or
% `struct' member, global or static variable, `typedef', or
% enumeration.  Identify the purpose of each in 25 words or less.
\subsubsection*{In frame.c:}

\begin{verbatim}
struct frame
{
  struct hash_elem frame_elem; /* For placing frames inside frame_tables. */
  struct list_elem eviction_elem; /* For placing frames into eviction_queue. */
  bool pinned;  /* Indicates whether frame is pinned (cannot be evicted) */
  uint32_t *pd; /* The owner thread's page directory. */
  /* Pointer to the page's entry in the supplementary page table. */
  struct supp_page_mapped *mapped;
  void *kpage;  /* The kernel virtual address of the frame. */
};
\end{verbatim}
Frame meta-data.
A frame is a page-aligned, page-sized physical storage unit in memory.
Pages are stored in frames, although their virtual addresses may differ from actual physical address.

\begin{verbatim}
struct frame_table
{
  struct lock table_lock; /* Synchronizes table between threads. */
  struct hash allocated;  /* Frames which have been allocated already. */
  /* Mirrors the above hash. Organises frames for page eviction. */
  struct list eviction_queue;
  /* Condition used to wait for a frame to be either unpinned or freed. */
  struct condition wait_table_changes;
  unsigned pinned_frames; /* Count of currently pinned frames. */
};
\end{verbatim}
The frame table keeps track of frames that are currently in use.
It uses redundant data-structures to provide quick look-up (hash table) alongside efficient eviction (queue).


\begin{verbatim}
static struct frame_table frames;
\end{verbatim}
Internal variable for keeping track of all allocated frames without exposing details to the rest of the OS.

%\subsubsection*{In swap.c:}
%TODO: Insert swap.c things here.

\subsection{Frame Eviction}

% When a frame is required but none is free, some frame must be
% evicted.  Describe your code for choosing a frame to evict.
Our code implements ``second chance'' page eviction algorithm.
When no free pages are available, we retrieve a list of all frames and look at its head.
If the access bit of the page in the head frame is set or the frame is pinned, we reset the access bit and reinsert that page into the back of the list.
We continue cycling through the list until we find a frame that is not pinned and whose page's access bit is not set and we evict this page.

Note that if every frame in the list is pinned, instead of iterating endlessly through it, we wait until a frame is either freed or unpinned.
In order to make this algorithm more efficient, we maintain a list of all used frames alongside the hash table that makes it possible to quickly find frame structures corresponding to a kernel page.

\subsection{Algorithms - What happens to Q when P evicts Q's frame}

% When a process P obtains a frame that was previously used by a
% process Q, how do you adjust the page table (and any other data
% structures) to reflect the frame Q no longer has?
When P selects Q's frame for eviction, it first acquires a lock over that frame's supplementary page table entry.
The evicted page is either written to swap or, if it represents a memory mapped file, back to disk.
The lock ensures that Q cannot page-fault its page back in before P is done writing that page.
In the former case, we add the number of the swap slot used to the frame's supplementary page entry.
Finally, P releases the lock over the page's supplementary table entry.

\subsection{Algorithms - Stack Growth Heuristic}

% Explain your heuristic for deciding whether a page fault for an
% invalid virtual address should cause the stack to be extended into
% the page that faulted.

We calculate the difference between the stack pointer and the invalid virtual
address, and see if it is larger than some limit. If it exceeds this limit, then
the user program is deemed to be buggy, and we terminate the thread.

This limit is defined to be 64 bytes, which is two word sizes, and is probably
sufficient as a limit.  Since we do not know every 80x86 assembly instruction,
we simply set the limit to be a large number within reason, instead of spending
a lot of time trying to find the smallest number that would work for the entire
instruction set. We did not want to make the number too large, or risk allowing
buggy user programs.

\subsection{Synchronization - Design}

% Explain the basics of your VM synchronization design.  In
% particular, explain how it prevents deadlock.  (Refer to the
% textbook for an explanation of the necessary conditions for
% deadlock.)
%TODO
swap - frame table:
  -write: swap table is always accessed from frame table + swap table synched internally
  -read: unsynchronised: can read concurrently %are you sure, Alan?
 											   %but are you sure sure?
 											   %I mean your retrieves lead to removal
 											   %can you even be reading the same slot?
frame table - supp table:
  -request: receives info from supp + internal synch
  -eviction: locks the supp page entry to prevent race
  -free: page removal is done process side
supp - swap table/disk
  -retrieve upon page fault:
    -disk: filesys\_lock
    -swap table: internal synch
  -free slots and pages

\subsection{Synchronization - Concurrent Accesses during page eviction}

% A page fault in process P can cause another process Q's frame
% to be evicted.  How do you ensure that Q cannot access or modify
% the page during the eviction process?  How do you avoid a race
% between P evicting Q's frame and Q faulting the page back in?
In order to prevent Q from accessing its page, during eviction we begin by removing the page from the page table.
We also acquire a lock to prevent process Q from attempting to fault its page back in before process P finishes writing the page to either swap or the file-system, if the page was a memory mapped file).

Race conditions related to the frame table itself are avoided through internal synchronisation: each external function in frame.c implicitly acquires a lock over the frame table before accessing it.
Hence, in this case, process Q will be blocked until process P finishes evicting the page.

\subsection{Synchronization - Concurrent eviction when reading in page}

% Suppose a page fault in process P causes a page to be read from
% the file system or swap.  How do you ensure that a second process Q
% cannot interfere by e.g. attempting to evict the frame while it is
% still being read in?
A page returned by the request\_frame function (the only way to acquire a page) is pinned by default.
In our implementation, the eviction algorithm cannot evict pinned pages, which allows the caller to finish processing the newly acquired page before unpinning it, thus making its eviction possible.

\subsection{Synchronization - Handling accesses to paged-out pages in syscalls}

% Explain how you handle access to paged-out pages that occur
% during system calls.  Do you use page faults to bring in pages (as
% in user programs), or do you have a mechanism for "locking" frames
% into physical memory, or do you use some other design?  How do you
% gracefully handle attempted accesses to invalid virtual addresses?
We use page faults to bring in pages: after a few sanity checks on each pointer, we dereference it, causing a page-fault.
Upon a page fault, we check the pointer against the supplementary page table and provide a page based on the information in the table, either retrieving the data from swap or file, or creating a new zeroed-out page if the page in question was should be on the stack.

The sanity checks look for threads attempting to access invalid virtual addresses, which is apparent if we cannot find data about the segment that contains this virtual address. In this case, we terminate the offending thread by calling thread\_exit(), which frees that threads resources (including occupied pages and swap space).

\subsection{Rationale - Synchronization decisions}

% A single lock for the whole VM system would make
% synchronization easy, but limit parallelism.  On the other hand,
% using many locks complicates synchronization and raises the
% possibility for deadlock but allows for high parallelism.  Explain
% where your design falls along this continuum and why you chose to
% design it this way.
Our design is synchronised via multiple locks, in an attempt to allow as much parallelism as possible.
In order to make synchronisation simpler, we attempted to use implicit synchronisation as far as possible; \textit{e.g.} many external functions (particularly in frame.c and swap.c) acquire internal resources on their own instead of relying on the caller to acquire them.
We also tried to avoid excessive synchronisation, that is, synchronising processes that do not require synchronisation. For instance, it is possible to read concurrently from the swap table (although writes are still synchronised).
