\section{Memory Mapped Files}

\subsection{Data Structures}

% Copy here the declaration of each new or changed `struct' or
% `struct' member, global or static variable, `typedef', or
% enumeration.  Identify the purpose of each in 25 words or less.

\subsubsection*{vm/mapped\_files.h}

\begin{verbatim}
struct mapid
  {
    mapid\_t mapid;
    struct file *file;
    struct supp\_page\_segment *segment;
    struct hash\_elem elem;
  };
\end{verbatim}

struct mapid is a data structure that keeps track of data for a memory mapped
file.

\subsubsection*{userprog/process.h}

\begin{verbatim}
typedef struct process_info
  {
...
    /* Hash used to for mapping ids to files */
    struct hash mapped\_files;
    /* Provides unique map ids for process */
    unsigned mapid\_counter;
...
  };
\end{verbatim}

mapped\_files maps mapid\_t to mapid structs.

mapid\_counter is used to allocate unique mapids to each mapid struct.

\subsection{Algorithms - Integration}

% Describe how memory mapped files integrate into your virtual
% memory subsystem.  Explain how the page fault and eviction
% processes differ between swap pages and other pages.

Memory mapped files simply register the existence of a segment of memory that
corresponds to the mapped file. Thus, later on, the supplementary page table treats
this segment like any other segment that is read from a file, and lazily loads
the pages in as they page fault.

Swap pages and non-swap pages differ from the fact that when pages are swapped
out, their swap number is set. By default, the swap number is set to a reserved
value that indicates the page is not in swap. Non-swap pages will either have
the default swap number, or will not have a mapping in the supplementary page
table, if they have not been mapped previously.

Swap pages are retrieved from swap when they page fault. Whenever a frame is
evicted, its swap number is recorded in the supplementary page table if the
frame table places it in swap (as opposed to just abandoning the frame, when it
isn't dirty).

File pages (pages inside file segments) are read in from the correct offset into
the file. The offset and a pointer to the file are kept inside the data for the
segment they belong to.

Stack pages are just zeroed out on the first read.

\subsection{Algorithms - Overlaps}

% Explain how you determine whether a new file mapping overlaps
% any existing segment.

In syscall\_mmap, supp\_page\_lookup\_segment is called starting with the mapping address and increments an index and then checks every address + index*PGSIZE upto the max number of pages requested by the mapping of the file. If a segment is found then the mmap returns -1 because a mapping already exists at that address. If no segment is found then segments is created and files data is set to that segment.

\subsection{Rationale - Code duplication}

% Mappings created with "mmap" have similar semantics to those of
% data demand-paged from executables, except that "mmap" mappings are
% written back to their original files, not to swap.  This implies
% that much of their implementation can be shared. Explain why your
% implementation either does or does not share much of the code for
% the two situations.

Registering a memory-mapped segment in the supplementary page table uses the
same two functions as used for registering executable segments, so the code is
fully shared in this situation.

The supplementary page table treats both of these segments as 'file segments'
during all of its processes, so the code is fully shared for page faults and
eviction as well.

Writing pages of a memory-mapped segment back to file had to be treated as a
special case, so a boolean had to be recorded in the segment data for this.
