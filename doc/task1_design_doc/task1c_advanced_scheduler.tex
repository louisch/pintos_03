\section{Advanced Scheduler}
\subsection{Data Structures}
\paragraph{B1:} % (5 marks)
% Copy here the declaration of each new or changed `struct' or `struct' member,
% global or static variable, `typedef', or enumeration.  Identify the purpose of
% each in 25 words or less.

Listed here is a list of files that have new or changed `struct' or `struct'
members, global or static variable, `typedef', or enumeration.

\begin{description}

\item[(src/lib/kernel/fixed\_point.h)] This file provides fixed point
  operations.

\begin{verbatim}
typedef int fixed_point;
\end{verbatim}
  This is a typedef int which represents the type fixed\_point. The use
  of the typedef is to help with readability of code.

\item[(src/threads/mlfqs.c)] This file provides provides the bulk of the
  functionality for the advanced scheduler.

\begin{verbatim}
/* Estimates the average number of threads ready to run over the
   past minute. */
static fixed_point load_avg = 0;
\end{verbatim}

  This variable is used to store the value of load\_avg somewhere so that it may
  be fetched at any time for testing.

\begin{verbatim}
static struct list ready_array[PRI_NUM];
\end{verbatim}

  ready\_array is an array of queues of threads. Allows lookup of threads by
  priority. PRI\_NUM is PRI\_MAX - PRI\_MIN.

\item[(src/threads/thread.h)]
  A diff of the changes made to struct thread:

\begin{verbatim}
@@ -88,11 +90,15 @@ struct thread
     char name[16];                      /* Name (for debugging purposes). */
     uint8_t *stack;                     /* Saved stack pointer. */
     int priority;                       /* Priority. */
+    int nice;                           /* Niceness. */
+    fixed_point recent_cpu;             /* CPU time this received 'recently' */
     struct list_elem allelem;           /* List element for all threads list. */

     /* Shared between thread.c and synch.c. */
     struct list_elem elem;              /* List element. */

+    struct list_elem mlfqs_elem;        /* Used by the MLFQS */
+
 #ifdef USERPROG
     /* Owned by userprog/process.c. */
     uint32_t *pagedir;                  /* Page directory. *
\end{verbatim}

  \begin{description}

  \item[int nice] A factor which determines the CPU time a thread takes or gives
    out to other threads. Threads with higher nice decays in priority faster.

  \item[fixed\_point recent\_cpu] The amount of CPU time this thread has received
    recently.

  \item[struct list\_elem mlfqs\_elem] Allows threads to be placed inside
    ready\_array.

  \end{description}

\end{description}

\subsection{Algorithms}
\paragraph{B2:} % (5 marks)
% Suppose threads A, B, and C have nice values 0, 1, and 2.  Each has a
% recent\_cpu value of 0.  Fill in the table below showing the scheduling
% decision and the priority and recent\_cpu values for each thread after each
% given number of timer ticks:

Assuming that no other threads apart from these three exist, so that only these
three are ever considered for running.

\begin{tabular}{ | l | r r r | r r r | l | }
  \hline
  \multirow{2}{*}{timer ticks} & \multicolumn{3}{|c|}{recent\_cpu} &
    \multicolumn{3}{|c|}{priority} & \multirow{2}{*}{thread to run} \\
     & A  & B & C & A  & B  & C  &   \\
  \hline
  0  & 0  & 1  & 2 & 63 & 60 & 58 & A                     \\
  4  & 4  & 1  & 2 & 62 & 60 & 58 & A                     \\
  8  & 8  & 1  & 2 & 61 & 60 & 58 & A                     \\
  12 & 12 & 1  & 2 & 60 & 60 & 58 & A \and B              \\
  16 & 14 & 3  & 2 & 59 & 60 & 58 & B                     \\
  20 & 14 & 7  & 2 & 59 & 59 & 58 & A \and B              \\
  24 & 16 & 9  & 2 & 59 & 58 & 58 & A                     \\
  28 & 20 & 9  & 2 & 58 & 58 & 58 & C, B, A, then C again \\
  32 & 21 & 10 & 4 & 57 & 58 & 58 & B \and C              \\
  36 & 21 & 12 & 6 & 57 & 58 & 57 & B                     \\
  \hline
\end{tabular}

\paragraph{B3:} % (5 marks)
% Did any ambiguities in the scheduler specification make values in the table
% uncertain?  If so, what rule did you use to resolve them?  Does this match the
% behaviour of your scheduler?

The order in which threads are chosen from the highest priority queue is
unspecified, and furthermore, the frequency with which the current thread should
be yielded is also unspecified.

The rules we used to resolve these ambiguities are the same as the ones that we
implemented in our scheduler.

We decided to simply add threads to the end of their respective queue when
yielding or unblocking, and to pop the first thread when fetching the next
thread to run. This FIFO strategy ensures we rotate through all threads in the
highest priority queue.

As for the frequency of yields, we simply yield every tick.

\paragraph{B4:} % (5 marks)
% How is the way you divided the cost of scheduling between code inside and
% outside interrupt context likely to affect performance?

We did all the calculations inside the interrupt handler. Inserting and popping
from the ready\_array is done outside, but moving of threads between queues in
the ready\_array is done inside the interrupt handler, as this needs to happen
for each ready thread if its priority is changed.

Since most of the work was done inside the interrupt handler, performance may be
affected adversely, as each tick necessarily has to be slowed down. However,
since calculations are mostly simple arithmetic operations, and because the
moving of a list\_elem from one list to another is fairly quick, this should not
be a large problem in the majority of cases.

Unfortunately, since the number of list\_elems which need to be moved is equal
to the number of ready threads in the worst case, the tick has complexity O(n)
where n is the number of ready threads. This is not terrible, especially
considering the actual update only happens every 4 ticks, but may be
unacceptable in larger OSs.

\subsection{Rationale}
\paragraph{B5:} % (5 marks)
% Briefly critique your design, pointing out advantages and disadvantages in
% your design choices.  If you were to have extra time to work on this part of
% the task, how might you choose to refine or improve your design?

We chose to place all ready threads into the array of queues, ready\_array, which forms a lookup table of sorts for the threads by priority. The advantage of using the ready\_array 
was that it allowed for constant time insertion of threads and n time sorting. However this comes with the disadvantage that it takes a constant space of 64*mlfqs\_elem as opposed to a
varying size ready\_list. 

For the sake of performance, we chose to make the fixed point operations macros,
with no type checking. This means one could potentially convert an integer to a
fixed point twice, as a fixed\_point is simply an alias for int.

We factored out many functions for interacting with ready\_array, as well as
ready\_array itself, into a separate file, mlfqs.c. This has many advantages,
apart from modularisation. It allows for encapsulation, hiding away functions
that are irrelevant to the operation of the scheduler, and preventing direct
modification of the ready\_array.

While we did try to create functions for representing various `units of work'
(for example, `calculate a new load\_avg'), and this does help with readability
of the code, some of the functions simply modify the global ready\_array
directly, rather than taking it as an argument. This couples the function with
the variable quite tightly, and would make further modularisation difficult.

If we had more time, we would probably change the functions to take arguments
such that the function could easily be modularised into different files if
desired.

\paragraph{B6:} % (5 marks)
% The assignment explains arithmetic for fixed-point mathematics in detail, but
% it leaves it open to you to implement it.  Why did you decide to implement it
% the way you did?  If you created an abstraction layer for fixed-point
% mathematics, that is, an abstract data type and/or a set of functions or
% macros to manipulate fixed-point numbers, why did you do so?  If not, why not?
